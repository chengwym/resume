% !TEX TS-program = xelatex
% !TEX encoding = UTF-8 Unicode
% !Mode:: "TeX:UTF-8"

\documentclass{resume}
\usepackage{zh_CN-Adobefonts_external} % Simplified Chinese Support using external fonts (./fonts/zh_CN-Adobe/)
% \usepackage{NotoSansSC_external}
% \usepackage{NotoSerifCJKsc_external}
% \usepackage{zh_CN-Adobefonts_internal} % Simplified Chinese Support using system fonts
\usepackage{linespacing_fix} % disable extra space before next section
\usepackage{cite}

\begin{document}
\pagenumbering{gobble} % suppress displaying page number

\name{Jingbo Cheng 程靖博}

\basicInfo{
  \email{chengwym@gmail.com} \textperiodcentered\ 
  \phone{(+86) 18686370855} \textperiodcentered\ 
  \github[github]{https://github.com/chengwym} \textperiodcentered\ 
  \linkedin[linkedin]{https://www.linkedin.com/in/jingbo-cheng-a661a3252/}}
 
\section{\faGraduationCap\  教育背景}
\datedsubsection{\textbf{北京大学}}{2021 -- 2025}
\textit{本科.} 计算机系

\section{\faHeartO\ 获奖情况}
\datedline{37届全国物理竞赛(CPhO)银牌}{2020年11月}

\section{\faUsers\ 实习经历}
\datedsubsection{\textbf{概率投资} \textit{机器学习实习生}}{2022年12月 -- 2022年3月}
模块化自动训练框架
\begin{itemize}
  \item 搭建完整训练框架,实验分为特征选择,模型训练,预测结果生成报告。 核心文件为配置文件,只需要修改配置文件内的一些参数,即可批量跑出结果对比。
  \item 使用numba,CPython等加速手段,使回测时间有所提升。
\end{itemize}
模型融合
\begin{itemize}
  \item 低频场景下,使用不同的y,不同的loss分别训练出不同的模型,再将训练出的模型信号用多种方法融合。
使用xgboost对信号融合,并增加了信号的可解释性。融合后产生的信号。在回测效果中一年扣除手续费后指数增强
年化超额收益有42\%,较融合前信号结果有较大提升。
\end{itemize}

\section{\faObjectGroup\ 项目经历}
\datedsubsection{\textbf{利用计算机视觉方法研究地震波形图} \textit{北京大学}}{2022年12月 -- 2023年1月}
\begin{itemize}
  \item 处理mseed格式数据,得到横坐标为offset,纵坐标为时间的地震图谱。
用resnet模型进行地震图谱的分类。
  \item 用resnet模型,通过地震图谱,预测地震的震源深度,冰川崩塌的力方位角。以及产生地点的经纬度。
\end{itemize}

\datedsubsection{\textbf{kaggle预测信用卡评分} \textit{北京大学}}{2022年5月 -- 2022年6月}
\begin{itemize}
  \item 基于梯度提升决策树的信用卡评分模型,以 Kaggle 公开数据集为基础, 进行了精细的特征工程和严格的统计分
  析, 选择了梯度提升决策树作为算法, 并建立有经济学意义的模型转换信用分。
  \item 选择梯度提升决策树模型, 采用 Baysian Optimization 进行超参数优化, 交叉验证选择训练轮数, 然后正式训练模
  型, 计算特征重要性——根据决策树的结点分裂情况计算。
  \item 在违约概率预测上得到了 0.86910 的优秀 AUC,与排行榜前五同样score。
\end{itemize}

% Reference Test
%\datedsubsection{\textbf{Paper Title\cite{zaharia2012resilient}}}{May. 2015}
%An xxx optimized for xxx\cite{verma2015large}
%\begin{itemize}
%  \item main contribution
%\end{itemize}

\section{\faCogs\ 技能}
% increase linespacing [parsep=0.5ex]
\begin{itemize}[parsep=0.5ex]
  \item C++, Python, PyTorch, Linux, LaTeX, Shell Script, HTML, JavaScript, Java
  \item CET4, CET6
\end{itemize}

%% Reference
%\newpage
%\bibliographystyle{IEEETran}
%\bibliography{mycite}
\end{document}
